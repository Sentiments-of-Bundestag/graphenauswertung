\section{Implementierung}

\subsection{Umsetzung}
Die Backend-Anwendung wurde als Python Flask Applikation entwickelt. Der gesamte Quellcode ist im GitHub Repository \textit{Sentiments-of-Bundestag/ Graphenauswertung}~\cite{github_graphenauswertung} verfügbar.

Die Applikation wird durch Ausführen der Datei \lstinline{app.py} gestartet.
Diese stellt zunächst die Verbindung mit den beiden Neo4j Datenbanken her.
Zur Kommunikation mit den beiden Datenbanken existiert jeweils eine Klasse, die den Zugriff auf die jeweilige Datenbank verwaltet und Methoden zur Abfrage der Inhalte der Datenbank bereitstellt.
Die Zugangsdaten für die Datenbanken werden als Umgebungsvariablen bereitgestellt. Diese werden beim Start der Applikation aus der Datei \lstinline{.env} eingelesen.
Aus Sicherheitsgründen, ist diese Datei nicht Teil der Quellcodeverwaltung.
Einige komplexere Prozeduren wie bspw. die Berechnug des PageRank wurden ausgelagert und werden von den Datenbankklassen importiert.
Die Defintion der Schnittstellen erfolgt in Form von Methoden, die mit dem Pfad annotiert sind, unter dem die jeweilige Schnittstelle erreichbar ist.
Aufgrund der großen Datenmenge und der damit verbundenen hohen Laufzeit werden alle Anfragen für 12 Stunden gecached.
Dazu wird die Bibliothek Flask-Caching mit dem Cache Typ SimpleCache verwendet, welcher einem In-Memory Python Dictionary entspricht.\cite{flask_caching}
Die Implementierung der Schnittstellenmethoden besteht in der Regel aus einem Aufruf einer Methode der Datenbankklassen, wobei das Ergebnis zur Übertragung in der HTTP Response in das JSON Format umgewandelt wird.

Die Methoden der Datenbankklassen sind alle nach dem gleichen Muster aufgebaut.
Die nachfolgende Methode, welche die verfügbaren Fraktionen abfragt, kann als Beispiel dienen.

\begin{lstlisting}[caption={Zugriff auf Neo4j DB: get\_factions}, captionpos=b]
def get_factions(self,
                 sentiment_type="NEUTRAL",
                 session_id=None):

    where = self.get_where_clause(sentiment_type, session_id)
    with self.driver.session() as session:
        query = \
            "MATCH (sender:{0})-[r:{1}]-(recipient:{0}) " \
            "{2} " \
            "with sender, " \
            "collect(distinct r.sessionId) as sesList, " \
            "collect(r) as rlist unwind rlist as r " \
            "RETURN DISTINCT sender.name as name, " \
            "sender.size as size, " \
            "sender.factionId as factionId, " \
            "sesList as sessionIds" \
            .format(NODE_FACTION, REL_COMMENTED, where)

    factions = session.run(query)
    return factions.data()
\end{lstlisting}

Die Methoden sind stets so aufgebaut, dass zunächst eine Datenbanksitzung geöffnet wird.
Zur Datenbankabfrage wird eine Cypher Query konstruiert. Diese hängt von den Parametern der HTTP-Anfrage ab, welche in der \lstinline{WHERE} Klausel der Query abgebildet werden.
Da viele Methoden die gleichen Parameter unterstützen, ist die Konstruktion der \lstinline{WHERE} Klausel in eine eigene Methode ausgelagert.
Die Query wird anschließend an die jeweilige Neo4j Datenbank geschickt. Das Resultat wird in einigen Fällen noch weiter verarbeitet und dann als Resultat zurückgegeben.
Mit dem Abschluss der Methode wird auch die Datenbanksitzung beendet.

Eine Herausforderung bei der Anbindung der Datenbanken bestand darin, dass beide zwar auf Neo4j basieren, jedoch wurde die Beziehung zwischen zwei Knoten (Abgeordnete bzw. Fraktionen) unterschiedlich abgebildet.
Im Graphen der Abgeordneten wurde eine Nachricht zwischen zwei Knoten als zusätzlicher Knoten abgebildet, während im Graphen der Fraktionen eine Nachricht als Beziehung implementiert wurde.
Die Arbeit mit der Datenbank der Abgeordneten wurde uns dadurch erleichtert, dass die verantwortliche Gruppe uns beispielhafte Cypher Queries zur Verfügung stellte.

\subsection{Bereistellung}
Zur Bereitstellung der Anwendung auf einem Server wurde ein Docker Image erstellt, mithilfe dessen die Anwendung gestartet werden kann.
Dabei wurde wie in \cite{flask_docker} beschrieben vorgegangen.
Das Docker Image basiert auf dem Standard Docker Image für Python und wird gebaut, indem zunächst die Bibliotheken, von denen die Anwendung abhängig ist installiert werden.
Anschließend wird der Quellcode und die Konfigurationsdateien in das Docker Image kopiert.
Zum Start der Anwendung wird ein Shell Skript aufgerufen, welches den uWSGI Applikationsserver startet.
HTTP-Anfragen auf dem Standard Port 80 werden mithilfe von nginx an den uWSGI Server weitergeleitet.
Um den Start der Anwendung ohne Angabe des Port Mappings zu gestalten wurde außerdem eine \lstinline{docker-compose} Datei erstellt.


Zur Bereistellung auf dem HTW Server (http://infosys6.f4.htw-berlin.de/) wurde Port 80 in der Firewall geöffnet und git und Docker installiert.
Der Start der neuesten Version der Anwendung auf dem Server erfolgt durch folgende Schritte:
\begin{itemize}
  \item Zugriff auf den Server via \lstinline{ssh}
  \item Herunterladen der neuesten Version des Quellcodes mithilfe von \lstinline{git clone} bzw. \lstinline{git pull}
  \item Erstellung des neuen Docker Images mithilfe von \lstinline{docker-compose build}
  \item Start der Applikation mithilfe von \lstinline{docker-compose up}
\end{itemize}

Die verfügbaren Schnittstellen wurden in der \lstinline{README} des Repositories beschrieben und so der nachfolgenden Gruppe (Nutzeroberfläche) zugänglich gemacht.
Die Nutzeroberfläche verwendet die Schnittstellen seither als Datengrundlage für sämtliche Diagramme auf der Website.
