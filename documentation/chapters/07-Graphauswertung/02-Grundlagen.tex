\section{Grundlagen}
\label{sec:7-grundlagen}

In diesem Kapitel werden die notwendigen Kenntnisse für das Verständnis der nachfolgenden Kapitel vermittelt. Zuerst wird der PageRank-Algorithmus erläutert mit Fokus auf eine effizientere Berechnung der Ränge mithilfe von Matrizenmultiplikationen und Eigenvektoren. Nachfolgend wird auf die wichtigsten verwendeten Entwicklungsframeworks, -Tools und -Konzepte eingegangen.

\subsection{PageRank}
Der PageRank-Algorithmus ist ein Algorithmus zur Gewichtung von Knoten innerhalb eines Netzwerks anhand der Anzahl ihrer eingehenden Beziehungen. Dabei geht in die Berechnung des PageRanks eines Knoten, neben der Anzahl an eingehenden Beziehungen, ebenso der PageRank der auf ihn verweisenden Knoten mit ein. 

Aufgrund der daraus entstehenden direkten Abhängigkeit des PageRank eines Knoten von den PageRanks der auf ihn verweisenden Knoten, ist vor allem bei großen Netzwerken ein genaue Berechnung aller PageRanks nicht immer in absehbarer Zeit möglich. Aus diesem Grund werden die Werte für die PageRanks bei der Berechnung meist iterativ angenähert. Dabei wird allen Knoten ein einheitlicher Startwert als PageRank vergeben. Meist wird als Startwert der Wert $1/N$ verwendet, wobei $N$ die Anzahl aller Knoten des Netzwerks ist. Danach wird der PageRank für alle Knoten mehrfach berechnet und so die Werte iterativ angenähert.

Eine Möglichkeit dieser iterativen Annäherung ist ein mehrfaches Iterieren über das gesamte Netzwerk und die rekursive Berechnung des PageRanks für jeden Knoten. Da im vorliegenden Anwendungskontext mit mehreren hundert Knoten und zig Tausend von Beziehungen zwischen den Knoten gerechnet werden kann, wäre das Iterieren über alle Knoten und Beziehungen nicht sehr effizient. Stattdessen können allerdings Matrizenmultiplikationen und eine Eigenvektorberechnung für die PageRank-Berechnung verwendet werden.

Das Netzwerk, welches aus mathematischer Sicht als gerichteter Graph betrachtet werden kann, kann als eine quadratische, stochastische Matrix abgebildet werden \cite{pagerank_eigenvector}. Stochastisch bedeutet in diesem Kontext, dass die Summen der Spalten der Matrix alle 1 betragen \cite{pagerank_eigenvector}. Wenn im Graphen von einem Knoten $j$ eine Beziehung  zum Knoten $i$ besteht, dann wird in der Matrix an der Stelle $ij$ der Wert $1/d^{+}(j)$ eingetragen, wobei $d^{+}(j)$ für den Ausgangsgrad von $j$ steht \cite{pagerank_eigenvector}. Wenn keine Beziehung besteht, wird an der Stelle $ij$ eine 0 eingetragen \cite{pagerank_eigenvector}. Für den Fall, dass ein Knoten gar keine ausgehenden Beziehungen besitzt, wird in seiner gesamten Spalte in der Matrix der Wert $1/N$ eingetragen \cite{pagerank_eigenvector}. Dadurch wird verhindert, dass der PageRank in Sackgassen-Knoten gewissermaßen ``versickert''. 

Wurde der Graph nun als Matrix abgebildet, so kann der PageRank iterativ mithilfe von Matrixmultiplikationen berechnet werden. Dabei wird der PageRank als Vektor der Länge $N$ dargestellt, meist befüllt mit dem Startwert $1/N$ für alle Knoten. Statt einer gesamten Iteration über das Netzwerk und den rekursiven Berechnungen neuer PageRanks für alle Knoten, muss nun nur eine simple Multiplikation der Matrix mit dem Vektor durchgeführt werden. Das Ergebnis der Multiplikation ist wieder ein Vektor und dieser enthält die neuen PageRanks aller Knoten. Nach mehrfacher Multiplikationen der Matrix mit dem PageRank-Vektoren ändern sich die Werte innerhalb des Vektor nicht mehr. Es wurde iterativ ein dominanter Eigenvektor der Matrix berechnet, welcher gleichzeitig den endgültigen PageRanks der Knoten des Netzwerks entspricht \cite{pagerank_eigenvector}. 

Diese iterative Berechnung des PageRanks mithilfe von Matrizenmultiplikationen setzt voraus, dass der zugrundeliegende gerichtete Graph stark zusammenhängend ist \cite{pagerank_eigenvector}. Um dies sicherzustellen sollte zu der Matrix, welche den Graphen abbildet, ein sogenannter Dämpfungsfaktor addiert werden \cite{pagerank_eigenvector}, meist in der Größenordnung von 0,15 bis 0,25. 

\subsection{Entwicklungsframeworks, -Tools und -Konzepte}